\nonstopmode

\documentclass{beamer}

\usepackage{euler}
\usepackage[no-math]{fontspec}
\usepackage{xltxtra}
\usepackage{wrapfig}

% 文章中に注釈を入れる
\usepackage{bibentry}
\nobibliography*

% 使っているフォントと入れかえること。
\setmainfont{VL PGothic}
\setsansfont{VL PGothic}
\setmonofont{VL Gothic}

% xetexについては以下が詳しい
% http://zrbabbler.sp.land.to/xelatex.html
\XeTeXlinebreaklocale "ja"

% navigation barをけして、ページ番号をいれる。
% ページ数をいれておくと、指摘してもらうときに便利
\usetheme{default}
\setbeamerfont{page number in head/foot}{size=\large}
\beamertemplatenavigationsymbolsempty
\setbeamertemplate{footline}[frame number]

% 言語によってはハイライトが効かなかったり、言語を使いわけたいときには以下のリンクを参照
% コメント等に日本語を使いたい場合にはjlistingを使う必要がある
% http://tex.stackexchange.com/questions/83882/how-to-highlight-python-syntax-in-latex-listings-lstinputlistings-command
\usepackage{listings}
\lstset{basicstyle=\ttfamily,
numbers=left,}


\title{Beamer サンプル}
\author{@Tomoki\_Imai}
\date[2014/04/30]

\begin{document}
\begin{frame}
  \titlepage
\end{frame}

\begin{frame}
  \frametitle{Python}
  \begin{figure}[htb]
  \centering
    \includegraphics[width=1.0\textwidth]{img/python.pdf}
    \caption{Pythonのロゴ}
  \end{figure}
\end{frame}

\begin{frame}
  \frametitle{コード}
  \lstinputlisting[caption=メイン関数]{main.py}
\end{frame}

\begin{frame}
  \bibentry{DBLP:books/aw/Knuth73a}
\end{frame}

\bibliographystyle{plain}
\bibliography{library.bib}

\end{document}


